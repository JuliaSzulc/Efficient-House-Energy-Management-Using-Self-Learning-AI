\documentclass{article}
\usepackage[T1]{fontenc}
\usepackage[polish]{babel}
\usepackage[utf8]{inputenc}
\usepackage{lmodern}
\usepackage{fullpage}
\usepackage{longtable}
\usepackage{tikz}
\usetikzlibrary{shapes,arrows,positioning}
\selectlanguage{polish}


\begin{document}
\begin{center}
\textbf{Projekt Zespołowy - Projekt zarządzania jakością – wymagania jakościowe}
\end{center}
\mbox{}\\
\paragraph{Założenia teoretyczne projektu zarządzania jakością}\mbox{}\\

Pojęcie jakości w projekcie postrzegane jest jako zgodność rezultatu projektu ze
specyfikacjami, przeznaczeniem i oczekiwaniami odbiorcy.
Zarządzanie jakością w projekcie obejmuje procesy zarządzania jakością jak i techniki których celem jest obniżenie ryzyka związanego z niedotrzymaniem wymogów przez końcowy efekt projektu. Proces zarządzania jakością w projekcie składa się z etapów:
\begin{itemize}
\item Planowania jakości
\item Zapewniania jakości
\item Kontroli jakości
\end{itemize}

\paragraph{Oczekiwania jakościowe odbiorcy projektu}\mbox{}\\

Odbiorca, czyli firma Samsung oczekuje od projektu, aby był on stworzony zgodnie z najwyższą jakością oraz w pełni przetestowany (zarówno jednostkowo jak i integracyjnie). Wyznacza również terminy realizacji kolejnych wersji systemu, po których następuje faza weryfikacji i akceptacji według zdefiniowanych kryteriów.

\paragraph{Wymagania jakościowe dla każdej z ról w projekcie}\mbox{}\\
    
Każdy członek zespołu pełni rolę odbiorcy prac od innych członków zespołu jak i dostarczyciela wysokiej jakości produktu dla współpracowników, którzy przejmują jego pracę. Na każdym z etapów tworzenia systemu, każdy z członków zespołu powinien informować o statusie wykonywania pracy oraz o ewentualnych problemach w trakcie ich wykonywania.

Lider zespołu odpowiada za wdrożenie i utrzymanie zarządzania jakością w projekcie jak i raportowaniem danych związanych ze wskaźnikami jakościowymi. Jest on odpowiedzialny również za techniczną spójność dokumentów, ich zgodność ze standardami jakości oraz za zapewnienie wysokiej jakości produktów.

Osoby do których skierowany jest projekt są odpowiedzialne za ustalanie kierunku prac, decyzje w sprawie ogólnych celów projektu.

Zadaniem osoby z ramienia uczelni (w przypadku naszego projektu, prof. Michał Woźniak) jest odpowiedzialna za odbiór i weryfikację tworzonych dokumentów projektowych oraz przestrzeganie terminów weryfikacji i akceptacji.

\paragraph{Wymagania jakościowe dla zadań w projekcie}\mbox{}\\

W projekcie w celu utrzymania wysokiej jakości systemu zostały zdefiniowane reguły akceptacji na każdym z etapów pracy projektowej. 

Dla zadań związanych z tworzeniem dokumentacji, po przygotowaniu danego etapu, każdy z członków zespołu musi się z nim zapoznać, oraz zgłosić ewentualne uwagi. Następnie zadanie to jest przekazywane do prof. Woźniaka w celu weryfikacji, następnie, gdy zostaną zgłoszone uwagi, zadanie to trafia do poprawy i po poprawie, zadanie jest wysyłane i uznawane za zakończone. 
Każde zadanie związane z tworzeniem kodu systemu również posiada kryteria jakości, takie jak testy jednostkowe oraz manualne sprawdzenie działania stworzonej w ramach tego zadania funkcjonalności. Następnie taki kod przechodzi przez etap Code Review przez pozostałych członków zespołu, nieuczestniczących bezpośrednio przy danym zadaniu.
Regularnie odbywają się również testy integracyjne całego systemu, aby mieć pewność, że działa on bez zarzutów.
\mbox{}\\\mbox{}\\\mbox{}\\
W projekcie został stworzony również dokument odpowiedzialny za kod źródłowy systemu. Opisane są w~nim szczegółowo konwencje na temat
\begin{itemize}
\item formatowania kodu
\item tworzenia komentarzy
\item tworzenia dokumentacji
\item projektowania i tworzenia testów
\item nazewnictwa zmiennych i metod
\item pracy z systemem kontroli wersji (nazewnictwo commitów i branchy)
\end{itemize}

\end{document}
