\documentclass{article}
\usepackage[T1]{fontenc}
\usepackage[polish]{babel}
\usepackage[utf8]{inputenc}
\usepackage{lmodern}
\usepackage{fullpage}
\selectlanguage{polish}



\begin{document}
\begin{center}
\textbf{Projekt Zespołowy - Zgłoszenie tematu projektu}
\end{center}

Zespół realizuje projekt we współpracy z firmą Samsung.

\paragraph{Skład zespołu}
\begin{itemize}
\item Filip Olszewski - lider zespołu - Numer Indeksu \textbf{226108}
\item Julia Szulc - Numer Indeksu \textbf{225939}
\item Dawid Czarneta - Numer Indeksu \textbf{226037}
\item Michał Popiel - Numer Indeksu \textbf{208351}
\item Jakub Frąckiewicz - Numer Indeksu \textbf{226033}
\end{itemize}
\paragraph{Samsung-end tech support}
Rafał Pilarczyk (r.pilarczyk@samsung.com)
\paragraph{Temat projektu}

\begin{center}
{\def\arraystretch{1.7}\tabcolsep=6pt
	\begin{tabular}{l p{10cm}} 
	Temat projektu& Zastosowanie algorytmów sztucznej inteligencji do podejmowania decyzji na podstawie danych sensorycznych (Reinforcement learning for decission process)\\
	Cele projektu& Minimalizacja zużycia energii w budynkach oraz zwiększenie satysfakcji i komfortu użytkowników budynku \\
	Adresat& Firma Samsung\\
	Interesariusze& prof. Michał Woźniak, Rafał Pilarczyk, Politechnika Wrocławska, Konferencja Projektów Zespołowych, firma Samsung, środowisko badaczy dziedziny Reinforcement Learning, firmy oferujące rozwiązania z zakresu inteligentnego domu i inteligentnej kontroli zużycia energii\\
	Użytkownicy& Projekt o charakterze badawczym. W przypadku komercyjnego zastosowania, użytkownicy to osoby prywatne - właściciele domów jednorodzinnych\\
	Ograniczenia projektu&Technologia wykonania - język Python3.5+, narzędzia PyCharm IDE oraz PyTorch. Czas trwania projektu - 3,5 miesiąca. Możliwy wymiar pracy członków zespołu wynoszący od czterech do sześciu godzin w tygodniu na osobę\\

\end{tabular}}
\end{center}






\end{document}
