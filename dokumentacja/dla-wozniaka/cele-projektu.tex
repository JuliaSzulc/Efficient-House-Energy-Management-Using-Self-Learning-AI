\documentclass{article}
\usepackage[T1]{fontenc}
\usepackage[polish]{babel}
\usepackage[utf8]{inputenc}
\usepackage{lmodern}
\usepackage{fullpage}
\selectlanguage{polish}



\begin{document}
\begin{center}
Projekt Zespołowy - Zgłoszenie tematu projektu - 5 marzec 2018
\end{center}

Zespół realizuje projekt we współpracy z firmą Samsung.

\paragraph{Skład zespołu}
\begin{itemize}
\item Filip Olszewski - lider zespołu - Numer Indeksu \textbf{226108}
\item Julia Szulc - Numer Indeksu \textbf{225939}
\item Dawid Czarneta - Numer Indeksu \textbf{226037}
\item Michał Popiel - Numer Indeksu \textbf{208351}
\item Jakub Frąckiewicz - Numer Indeksu \textbf{226033}
\end{itemize}

\paragraph{Temat projektu}

\begin{center}
{\def\arraystretch{1.5}\tabcolsep=6pt
	\begin{tabular}{l p{10cm}} 
	Temat projektu& Zastosowanie algorytmów sztucznej inteligencji do podejmowania decyzji na podstawie danych sensorycznych. (Reinforcement learning for decission process)\\
	Cel projektu& Opracowanie programu - agenta w inteligentny sposób zarządzającego zużyciem energii poprzez interakcję ze środowiskiem, oraz symulatora przykładowego środowiska (budynek parterowy)\\
	Adresat& Firma Samsung\\
	Interesariusze& prof. Mariusz Woźniak, Rafał Pilarczyk, Politechnika Wrocławska, Konferencja Projektów Zespołowych oraz firma Samsung\\
	Użytkownicy& Projekt o założeniu badawczym. W przypadku komercyjnego zastosowania, użytkownicy to osoby prywatne - właściciele domów jednorodzinnych\\
	Ograniczenia projektu&Technologia wykonania - język Python3.5+, narzędzia PyCharm IDE oraz PyTorch. Czas trwania projektu - 3,5 miesiąca.\\

\end{tabular}}
\end{center}


\paragraph{Samsung-end tech support}
Rafał Pilarczyk (r.pilarczyk@samsung.com)



\end{document}
